%! Author = ewan
%! Date = 10/6/25

% Preamble
\documentclass[11pt]{article}

% Packages
\usepackage{amsmath}
\usepackage{xcolor}
\usepackage{hyperref}

% Document
% Document
\begin{document}

    \title{
            { \textbf{Security Protocols and Verification}} \\[1ex]
        {\small Defense of Cryptographic Protocol}
    }


    \author{
        Garance Frolla \\
        Ely Marthouret \\
        Ewan Decima\\ \\
        Team: \textbf{ASKO OM8464A2}
    }

    \date{September / November 2025}


    \maketitle
    \tableofcontents
    \newpage

    \section{Defense against PROTOxyde d'alCOl}

    \subsection{Notation}
    \begin{itemize}
        \item Let $\mathcal{K}_I$ denote the set of all facts known to the intruder $I$.
        \item Let $\bigl(K_I\bigr)_{I \in \mathcal{I}}$ the set of key use by $I$ during the \textit{bruteforce} step.
        \item Let $\mathcal{M}$ denote the set of all  messages sent during a communication between to agents.
        \item Let $M_{1,A} \in \mathcal{M}$ denote the first message sends by Alice, i.e., $M_{1,A} = \{|\langle A,N_A \rangle|\}_{K_{AB}}$
        \item Let $s(\cdot, \cdot)$ denote the sender function, i.e., $s(m,x)$ means message $M$ is sent by agent $C$.
        \item Let $[\cdot]_{(\cdot)}$ denote the extract function of a tuple message, i.e., for
                $ M = \langle m_1, m2, ..., m_n \rangle \in \mathcal{M}$, $[M]_i = m_i \quad \forall i \in [|1,n|]$.
        \item Let $\langle X',Y',Z', \Sigma' \rangle$ denote four random value.
    \end{itemize}
    

    \subsection{The attack}
    \begin{itemize}
        \item First this our understanding of your attack : the intruder $I$ steal the first alice's message :
        $M_{1,A}$. At this step $ K(N_A) \notin \mathcal{K}_I  $ and $K\bigl(K_{AB}\bigr) \notin \mathcal{K}_I$.


        \item After that, $I$ impersonates S by crafting the ticket $\{|\langle A,\tau,\lambda, K_{AB}\rangle|\}_{K_{BS}}$,
                replacing $K_{BS}$ with keys $K_I$ to form $T_I := \{|\langle X',Y',Z', \Sigma' \rangle|\}_{K_{I}}$
            and sending $M_{1,A}$ and $T_I$ to $B$.
        
        \item $B$ gets $M_{1,A}$. At this point $K\bigl( S(M_{1,A},A) \bigr) \notin \mathcal{K}_B$.
                But it's normal according to the ASKO OM8464A2 protocol. Then $B$ gets the crafted ticket $T_I$. $B$
                will decipher it with $K_{BS}$ and send back to $[dec(T_I, K_{BS})]_1$
                \begin{center}
                    $\Bigl\{\Bigl|dec\bigl(\{|N_A + 1|\}_K_{AB},[dec(T_I, K_{BS})]_4\bigr) \Bigr|\Bigr\}_{[dec(T_I, K_{BS})]_4}$
                \end{center}
        For better understanding, let us denote $[dec(T_I, K_{BS})]_1$, $[dec(T_I, K_{BS})]_2$, $[dec(T_I, K_{BS})]_3$, $[dec(T_I, K_{BS})]_4$
        as $X,Y,Z,\Sigma$ respectively.


        \item The attack lies on the fact that identities are short bitstring,  $B$ will always decipher the ticket
                $T_I$ with his symmetric key $K_{BS}$, and hoping that:
                \begin{center}
                    $\exists \: J \in \mathcal{I} \: | J \neq BS \wedge \: dec(T_J, K_{BS}) = \langle A,Y,Z, \Sigma \rangle$.
                \end{center}
                With such a key $K_J$, $B$ will think that $\Sigma$ is $K_{AB}$. At this point $K\bigl(\Sigma \bigr) \notin \mathcal{K}_I$
                and $K\bigr( K_J \bigl) \notin \mathcal{K}_I$.

    \end{itemize}

    \subsection{Refutation}
    
    \section{Conclusion}

\end{document}