\section{Protocol Description}

\subsection{Messages cost and description}



This protocol begins with entity $A$ generating a nonce, denoted as $N_A$. $A$ then encrypts her identity together
with the nonce using the freshly generated session key $K_{AB}$. The transmitted data is structured as follows:
$\{| \langle A,N_A \rangle |\}_{K_{AB}}$. This message costs \textbf{63}.

\vspace{1cm}

After sending the first message, $A$ sends to the honest and trusted server $S$, using the shared key $K_{AS}$,
the identity of $B$, a timestamp $\tau$, a lifetime period $\lambda$ to confirm the key and the session key $K_{AB}$.
The transmitted data is structured as follows: $\{| \langle B, \tau, \lambda, K_{AB} \rangle |\}_{K_{AS}}$. This message costs \textbf{166}.

\vspace{1cm}

First, $S$ checks whether $t \geq \tau + \lambda$, where $t$ denotes the time at which $S$ receives the message from
$A$. If this condition holds, $S$ aborts. Otherwise, using the shared key $K_{BS}$, $S$ sends to $B$ essentially the
same message as before, except that $A$ is replaced with $B$. The transmitted data is structured as follows:
$\{| \langle A, \tau, \lambda, K_{AB} \rangle |\}_{K_{BS}}$. This message costs: \textbf{166}.

\vspace{1cm}

$B$ receives the message $\{| \langle A, \tau, \lambda, K_{AB} \rangle |\}_{K_{BS}}$ and obtains the session key $K_{AB}$.
He also learns the validity period $\lambda$, starting from time $\tau$, during which $A$ will accept his response.
This measure provides protection against ticket theft. Indeed, even if an attacker manages to intercept a ticket,
they will not be able to use it after its expiration. When $B$ receives this message at time $t$, if
$t \geq \tau + \lambda$, then $B$ aborts.

Otherwise, $B$ respond to the first message of $A$, he can decrypt $\{| \langle A, N_A \rangle |\}_{K_{AB}}$ with the session key.
$B$ verifies that the $A$ in $\{| \langle A, N_A \rangle |\}_{K_{AB}}$ is the same compared to the first message he can now decrypt : $\{| \langle A,N_A \rangle |\}_{K_{AB}}$.
Key confirmation lies in the fact that $B$ sends back $N_A + 1$ to $A$. In this way, $A$ knows that $B$ has successfully
retrieved the key. This allows combining key confirmation with the challenge–response mechanism for the authentication
of $B$ with respect to $A$. The transmitted data is structured as follows: $\{| N_A + 1|\}_{K_{AB}}$.
This message costs \textbf{12}.

\vspace{1cm}


At the end, when $A$ receives the last message from $B$ at time $t$, she checks whether $t \geq \tau + \lambda$. If
this condition holds, $A$ rejects the message and aborts. Otherwise, the key exchange protocol succeeds.
\vspace{1cm}

The total cost is: \textbf{409}.
