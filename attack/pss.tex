%! Author = garance
%! Date = 10/21/25

% Fichier LaTeX décrivant l'attaque (version française)
\documentclass[11pt]{article}

% Packages
\usepackage[french]{babel}
\usepackage[utf8]{inputenc}
\usepackage[T1]{fontenc}
\usepackage{amsmath}
\usepackage{xcolor}
\usepackage{hyperref}
\usepackage{geometry}
\geometry{margin=2.5cm}

\begin{document}


\title{\textbf{Protocoles de sécurité et vérification} \\
{\small \textit{attaque sur PSS}}}

\author{Garance Frolla \\ Ely Marthouret \\ Ewan Decima \\\\ \textbf{Équipe : ASKO OM8464A2}}
\date{Septembre / Novembre 2025}

\maketitle
\tableofcontents
\newpage

\section{Protocole étudié}

Votre protocole:

\begin{enumerate}
    \item $B \rightarrow A : \{N_B,B\}_{pk(A)}$
    \item $A \rightarrow B : \{K_{AB},A,N_A,N_B\}_{pk(B)}$
    \item $B \rightarrow A : \{|\{B,N_A -1\}_{pk(A}|\}_{K_{AB}}$
    \item $A \rightarrow B : \{|N_B -1|\}_{K_{AB}}$
\end{enumerate}

\bigskip
\noindent
L'intrus $I$ écoute les communications entre $A$ et $B$ et intercepte tous les termes. Avec le temps, on suppose que $I$ connaît une ancienne clé $K_{AB}^*$.

\section{Description synthétique de l'attaque}

\subsection{Hypothèses}
\begin{itemize}
    \item $I$ est capable d'intercepter et de relayer des messages 
    \item $I$ a obtenu, lors d'une session précédente, une ancienne clé $K_{AB}^*$.
    \item Les spécifications n'indiquent pas que les agents verifient la fraîcheur des Nonces à la réception. Les agents verifient uniquement que les Nonces finaux correspondent bien aux Nonces initiaux. Il n'est également pas vérifié que c'est une ancienne clé.

    

\end{itemize}

\subsection{Déroulement de l'attaque}

\textbf{Attaque : I rejoue une ancienne session en prenant le rôle de $B$. Les Nonces sont anciens et la clé $K_{AB}^*$ aussi}

\begin{enumerate}
    \item $I(B) \rightarrow A : \{\textcolor{red}{N_B^*},B\}_{pk(A)}$
    \item $A \rightarrow I(B) : \{\textcolor{red}{K_{AB}^*},A,\textcolor{red}{N_A^*},\textcolor{red}{N_B^*}\}_{pk(B)}$
    \item $I(B) \rightarrow A : \{|\{B,\textcolor{red}{N_A^* -1}\}_{pk(A}|\}_{\textcolor{red}{K_{AB}^*}}$
    \item $A \rightarrow I(B) : \{|\textcolor{red}{N_B^* -1}|\}_{\textcolor{red}{K_{AB}^*}}$
\end{enumerate}



\section{Conséquences et conclusion}

\noindent
À la fin de l'attaque, $B$ croit avoir établi une communication sécurisée avec A pour la session. En réalité, la clé utilisée est une clé ancienne $K_{AB}^*$ supposée compromise.

\begin{itemize}
    \item $B$ pense détenir une clé $K_{AB}$ fraîche fournie par $A$ mais a en fait accepté $K_{AB}^*$ transmis par $I$.
    \item $I$ peut lire toutes les communications ultérieures chiffrées avec $K_{AB}^*$.
\end{itemize}

\end{document}
