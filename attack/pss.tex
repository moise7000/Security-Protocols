%! Author = ewan
%! Date = 10/6/25

% Preamble
\documentclass[11pt]{article}

% Packages
\usepackage{amsmath}
\usepackage{xcolor}
\usepackage{hyperref}

% Document
% Document
\begin{document}

    \title{
            { \textbf{Security Protocols and Verification}} \\[1ex]
        {\small Attack of Cryptographic Protocols}
    }


    \author{
        Garance Frolla \\
        Ely Marthouret \\
        Ewan Decima\\ \\
        Team: \textbf{ASKO OM8464A2}
    }

    \date{September / November 2025}


    \maketitle
    \tableofcontents
    \newpage



    \section{Attack on PSS}

    \begin{enumerate}
        \item \textcolor{red}{$B \rightarrow I(A) : N_B$}
        \item $I \rightarrow A : N_I$
        \item $A \rightarrow I : \{K_{AI}, A, N_A, N_I\}_{pub(I)}$
        \item \textcolor{red}{$I(A) \rightarrow B : \{\textcolor{red}{K_{AI}}, A, N_A, N_B\}_{pub(B)}$}
        \item \textcolor{red}{$B \rightarrow I(A) : \{|\{N_B, N_A -1\}_{pub(A)}|\}_{K_{AI}}$}
        \item \textcolor{red}{$I(A) \rightarrow B : \{|N_B-1|\}_{K_{AI}}$}
        \item $I \rightarrow A : \{| \{N_B, N_A-1\}_{pub(A)} |\}_{K_{AI}}$
        \item $A \rightarrow I : \{N_B-1\}$
    \end{enumerate}

    \section{Attack Description}
    \subsection{Attack Flow}

    \begin{itemize}
        \item \textbf{Message 1:} $B$ initiates the protocol, believing they are establishing a session with $A$. $B$ sends their fresh nonce $N_B$ to what they think is $A$, but the intruder $I$ intercepts this message while impersonating $A$. The intruder stores $N_B$ for later use in the attack.

        \item \textbf{Message 2:} The intruder $I$ initiates a separate, parallel session with $A$. $I$ sends their own nonce $N_I$ to $A$, who believes they are receiving a legitimate protocol initiation from $I$.

        \item \textbf{Message 3:} $A$ responds by generating a session key $K_{AI}$ (intended for secure communication with $I$) and their own nonce $N_A$. $A$ encrypts the tuple $\{K_{AI}, A, N_A, N_I\}$ using $I$'s public key $pub(I)$ and sends it to $I$. Since $I$ possesses the corresponding private key, they can decrypt this message and obtain $K_{AI}$ and $N_A$.

        \item \textbf{Message 4:} This is the critical step of the attack. The intruder $I$ impersonates $A$ to $B$ by constructing a fraudulent message. $I$ reuses the session key $K_{AI}$ that $A$ generated for them, but substitutes the nonce $N_I$ with $B$'s original nonce $N_B$. The message $\{K_{AI}, A, N_A, N_B\}_{pub(B)}$ is encrypted with $B$'s public key, making it appear as a legitimate response from $A$ to $B$'s initial request. $B$ decrypts this message and believes that $A$ has established a shared session key with them.

        \item \textbf{Message 5:} $B$ continues the protocol by computing $N_A - 1$ and encrypting $\{N_B, N_A - 1\}$ with $A$'s public key, then encrypting the result again with what $B$ believes is the shared session key (actually $K_{AI}$). $B$ sends this double-encrypted message to what they think is $A$, but $I$ intercepts it.

        \item \textbf{Message 6:} The intruder $I$, still impersonating $A$ to $B$, completes the protocol with $B$ by computing $N_B - 1$ and sending it encrypted under $K_{AI}$. $B$ can verify this response and believes the protocol has completed successfully with $A$.

        \item \textbf{Message 7:} Meanwhile, $I$ forwards the double-encrypted message from Message 5 to the real $A$. Since the inner encryption uses $A$'s public key and the outer encryption uses $K_{AI}$ (which $A$ established with $I$), $A$ can decrypt it successfully.

        \item \textbf{Message 8:} $A$ completes their session with $I$ by computing $N_B - 1$ and sending it encrypted under $K_{AI}$. Note that $A$ believes they are completing the protocol with $I$, using nonce $N_B$ that $I$ had sent in Message 2.
    \end{itemize}

    \subsection{Attack Results}

    At the conclusion of this attack, both $A$ and $B$ believe they have successfully established a secure session with each other using the session key $K_{AI}$. However, this session key was actually generated by $A$ for communication with the intruder $I$. Since $I$ possesses this key, they can decrypt and read all subsequent messages exchanged between $A$ and $B$.

    More precisely:
    \begin{itemize}
        \item $A$ believes they completed the protocol with $I$ and that $K_{AI}$ is shared only with $I$
        \item $B$ believes they completed the protocol with $A$ and that $K_{AI}$ (which $B$ thinks is $K_{AB}$) is shared only with $A$
        \item In reality, $I$ knows $K_{AI}$ and can act as a man-in-the-middle, decrypting, reading, and potentially modifying all messages between $A$ and $B$
    \end{itemize}

    This is a classic man-in-the-middle attack that exploits the protocol's failure to adequately bind the session key to both participants' identities in a way that prevents such substitution attacks.



\end{document}