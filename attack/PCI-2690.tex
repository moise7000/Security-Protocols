%! Author = garance
%! Date = 10/21/25

% Fichier LaTeX décrivant l'attaque (version française)
\documentclass[11pt]{article}

% Packages
\usepackage[french]{babel}
\usepackage[utf8]{inputenc}
\usepackage[T1]{fontenc}
\usepackage{amsmath}
\usepackage{xcolor}
\usepackage{hyperref}
\usepackage{geometry}
\geometry{margin=2.5cm}

\begin{document}


\title{\textbf{Protocoles de sécurité et vérification} \\
{\small \textit{attaque sur PCI-2690}}}

\author{Garance Frolla \\ Ely Marthouret \\ Ewan Decima \\\\ \textbf{Équipe : ASKO OM8464A2}}
\date{Septembre / Novembre 2025}

\maketitle
\tableofcontents
\newpage


\section{Description synthétique de l'attaque}

L'attaquant $I$ écoute les communications entre $A$ et $B$ et intercepte tous les termes échangés. L'attaquant $I$ établit
un nombre conséquent de communications légitimes avec $A$ et avec $B$, $S$ étant sollicité deux fois plus, afin de
surcharger les caches de $A$, $B$ et $S$. On suppose que $I$ connaît une ancienne clé $K^*$ de session entre $A$ et $B$.
De plus, on suppose qu'au moment de l'attaque les caches des agents $A$ et $B$ ne contiennent plus les nonces
précédemment interceptés.

\subsection{Hypothèses}
\begin{itemize}
    \item $I$ est capable d'intercepter et de relayer des messages 
    \item $I$ a obtenu, lors d'une session précédente, une ancienne clé $K^*$.
    \item Les caches de $A$ et $B$ ne sont limités et suppriment leurs données au fur et à mesure.

    

\end{itemize}

\subsection{Déroulement de l'attaque}

$I$ intercepte une communication légitime entre $A$ et $B$. $I$ établit un nombre $x$ de communications avec $A$
(respectivement $B$) afin de surcharger son cache. On suppose que les caches des participants ont une capacité de
$\frac{x}{3}$. À partir de ce moment, le cache de $A$ (respectivement $B$, $S$) est dépourvu tous les nonces précédent
les communication initiées par $I$. Ensuite, $I$ se faisant passer pour $A$, rejoue la communication entre $A$ et $B$.





\section{Conséquences et conclusion}

\noindent
À la fin de l'attaque, $B$ croit avoir établi une communication sécurisée avec A. En réalité, la clé
utilisée est une ancienne clé ($K^*$) connue par $I$.

\begin{itemize}
    \item $B$ pense détenir une clé $K$ fraîche fournie par $A$ mais a en fait accepté $K^*$ transmis par $I$.
    \item $I$ peut lire toutes les communications chiffrées avec $K^*$.
\end{itemize}

\end{document}
