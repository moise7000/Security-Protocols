%! Author = garance
%! Date = 10/21/25

% Fichier LaTeX décrivant l'attaque (version française)
\documentclass[11pt]{article}

% Packages
\usepackage[french]{babel}
\usepackage[utf8]{inputenc}
\usepackage[T1]{fontenc}
\usepackage{amsmath}
\usepackage{xcolor}
\usepackage{hyperref}
\usepackage{geometry}
\geometry{margin=2.5cm}

\begin{document}


\title{\textbf{Protocoles de sécurité et vérification} \\
{\small \textit{attaque sur PCI-2690}}}

\author{Garance Frolla \\ Ely Marthouret \\ Ewan Decima \\\\ \textbf{Équipe : ASKO OM8464A2}}
\date{Septembre / Novembre 2025}

\maketitle
\tableofcontents
\newpage


\section{Description synthétique de l'attaque}

L'attaquant $I$ écoute les communications entre $A$ et $B$ et intercepte tous les termes. L'attaquant $I$ établit un nombre très conséquent de communication avec $A$ et $B$, ainsi $S$ est sollicité 2 fois plus que $A$ ou $B$. Avec le temps, on suppose que $I$ connaît une ancienne clé $K^*$ et que les caches des agents $A$ et $B$ sont pleins et donc qu'ils se vident au fur et à mesure.

\subsection{Hypothèses}
\begin{itemize}
    \item $I$ est capable d'intercepter et de relayer des messages 
    \item $I$ a obtenu, lors d'une session précédente, une ancienne clé $K^*$.
    \item Les caches de $A$ et $B$ ne sont pas infinis et se suppriment leurs données au fur et à mesure

    

\end{itemize}

\subsection{Déroulement de l'attaque}

\textbf{Attaque : I rejoue une ancienne session en prenant le rôle de $A$. Les Nonces sont anciens et la clé $K^*$ aussi mais $A$ et $B$ ont été suffisament sollicités pour que les anciens Nonces ne soient plus dans le cache des utilisateurs}



\section{Conséquences et conclusion}

\noindent
À la fin de l'attaque, $B$ croit avoir établi une communication sécurisée avec A pour la session. En réalité, la clé utilisée est une clé ancienne $K^*$ supposée compromise.

\begin{itemize}
    \item $B$ pense détenir une clé $K$ fraîche fournie par $A$ mais a en fait accepté $K^*$ transmis par $I$.
    \item $I$ peut lire toutes les communications ultérieures chiffrées avec $K^*$.
\end{itemize}

\end{document}
