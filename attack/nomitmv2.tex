%! Author = garance
%! Date = 10/21/25

% Fichier LaTeX décrivant l'attaque (version française)
\documentclass[11pt]{article}

% Packages
\usepackage[french]{babel}
\usepackage[utf8]{inputenc}
\usepackage[T1]{fontenc}
\usepackage{amsmath}
\usepackage{xcolor}
\usepackage{hyperref}
\usepackage{geometry}
\geometry{margin=2.5cm}

\begin{document}


\title{\textbf{Protocoles de sécurité et vérification} \\
{\small \textit{attaque sur NOMITM}}}

\author{Garance Frolla \\ Ely Marthouret \\ Ewan Decima \\\\ \textbf{Équipe : ASKO OM8464A2}}
\date{Septembre / Novembre 2025}

\maketitle
\tableofcontents
\newpage

\section{Protocole étudié}

Votre protocole:

\begin{enumerate}
    \item $A \rightarrow B : A,N_a$
    \item $B \rightarrow A : \{B,N_b,N_a\}_{pub(A)}$
    \item $A \rightarrow B : \{K_{AB},N_b,N_a,A\}_{pub(B)}$
    \item $B \rightarrow A : H(K_{AB},N_a)$
\end{enumerate}

\bigskip
\noindent
L'intrus $I$ écoute les communications entre $A$ et $B$ et intercepte tous les termes. Avec le temps, on suppose que $I$ connaît une ancienne clé $K^*$.

\section{Description synthétique de l'attaque}

\subsection{Hypothèses}
\begin{itemize}
    \item $I$ est capable d'intercepter et de relayer des messages 
    \item $I$ a obtenu, lors d'une session précédente, une ancienne clé $K^*$.
    \item Les spécifications indiquent "4. B vérifie les nonces". Cependant il n'est pas écrit que les agents verifient la fraîcheur des Nonces à la réception. Cela signifie que les agents verifient que les Nonces finaux correspondent bien aux Nonces initiaux. Il n'est également pas vérifié que c'est une ancienne clé.

\end{itemize}

\subsection{Déroulement de l'attaque}

\textbf{Attaque : I rejoue une ancienne session en prenant le rôle de $A$. Les Nonces sont anciens et la clé $K^*$ aussi}

\begin{enumerate}
    \item $I(A) \rightarrow B : A,\textcolor{red}{N_a^*}$
    \item $B \rightarrow I(A) : \{B,\textcolor{red}{N_b^*},\textcolor{red}{N_a^*}\}_{pub(A)}$
    \item $I(A) \rightarrow B : \{\textcolor{red}{K_{AB}},\textcolor{red}{N_b^*},\textcolor{red}{N_a^*},A\}_{pub(B)}$
    \item $B \rightarrow I(A) : H(\textcolor{red}{K_{AB}},\textcolor{red}{N_a^*})$
\end{enumerate}


\section{Conséquences et conclusion}

\noindent
À la fin de l'attaque, $B$ croit avoir établi une communication sécurisée avec A pour la session. En réalité, la clé utilisée est une clé ancienne $K^*$ supposée compromise.

\begin{itemize}
    \item $B$ pense détenir une clé $K$ fraîche fournie par $A$ mais a en fait accepté $K^*$ transmis par $I$.
    \item $I$ peut lire toutes les communications ultérieures chiffrées avec $K^*$.
\end{itemize}

\end{document}
